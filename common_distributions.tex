\section{Common Distributions}

\subsection{Common Discrete Distributions}

\paragraph{Bernoulli Distributions}
A Bernoulli trial is an experiment with two outcomes;
success and failure.
A random variable \(X\) is defined with
\[
    X = \begin{cases}
        1 & \text{if  success}, \\
        0 & \text{if  failure}.
    \end{cases}
\]

Let \(p\in[0, 1]\) be probability of cuess. Then, we denote
\(X\sim \text{Bernoulli}(p)\)
\begin{enumerate}
    \item \(\mathbb{P}(X = 1) = p\),
    \item \(\mathbb{P}(X=0)  = 1-p\),
    \item \(\mathbb{E}(X) = p\),
    \item \(\Var(X) = \mathbb{E}(X^2) - \mathbb{E}(X)^2 = 1\times p - p^2 =
    p(1-p)\).
\end{enumerate}

\paragraph{Binomial Distribution}
When there are \(n\) independent bernoulli trials with
a success rate of \(p\), and \(X \coloneqq\) total number of succeses.
Then, \(X\) is a Binomial r.v with parameter \(n\) and \(p\)
such taht we write \(X\sim \text{Bin}(n, p)\).

Let \((Y_i)_{i=1, \dots, n}\) be a sequence of independent
bernoulli trials with success rate \(p\)..
Then \[
    X \coloneqq \sum_{i=1}^n Y_i \text{ is } \text{Bin}(n, p).
\]

Expectation exists as \[
    \mathbb{E}(X)
    = \mathbb{E}\left(\sum_{i=1}^{n} Y_i\right)
    = \sum_{i=1}^{n} \mathbb{E}(Y_i) = np.
\]

Alternatively, using combinatorics,
\[
    \mathbb{P}(X=k) = C^{n}_{k} p^k (1-p)^{n-k}.
\]

\paragraph{Poisson Distribution}
A random variable \(X\) follows the poisson distribution
with parameter \(\lambda\) if its probability function is
\[
    \mathbb{P}(X = k)
    = \frac{\lambda^k e^{-\lambda}}{k!}.
\]

Observe that
\[
    \mathbb{E}(X) = \lambda.
\]

\subparagraph{Use} The poisson distribution is used to model
count data. That is, counting the number of times
an event occurs within a time period.
The parameter \(\lambda\) epresents the average number of times
the event occurs in the time period of interest.



\paragraph{Hypergeometric Distribution}
A random variable has hypergeometric distribution with
parameter \(N, m, n\) and is written as
\(X\sim \Hyp(n,m N)\) if
\[
    \mathbb{P}(X = k)
    =
    \frac{C_x^m \times C^{N-m}_{n-x}}{C^N_n},
    \quad
    \text{ where } x = 1,\dots, n.
\]
\subparagraph{Example of Hypergeometric}
Gien a box of \(N\) balls, \(m\) are red and
\(N - m\) are black. Draw \(n\) balls at random and let
\(X\) be the number of red balls drawn. Then, \(X\sum \Hyp(n, m, N)\).

\subparagraph{Remark}
The ``I feel like skipping this... discrete problems are not very interesting''
was stated while covering this. Interpretation of this
is left as an exercise to the reader.


\subsection{Continous Distributions}

\paragraph{Gaussian - Normal Random Variable}
Given parameters \(\mu, \sigma^2\) has a probability density
function as
\[
    f_X(x)
    =
    \dfrac{a}{\sqrt{2\pi\sigma^2}}
    e^{-\frac{(x-\mu)^2}{2\sigma^2}},
    \quad
    x\in \mathbb{R}.
\]
The expecation and variance are \(\mu, \sigma^2\)
respectively.

\paragraph{Linear Transforms}
Let \(X\) be a r.v with a probability
density function \(f_X\). Let
\(y = a + bX\) then for \(b > 0\) and \(a\in\mathbb{R}\),
then \[
    f_Y(x) = \frac{1}{b}f_X \left(
        \frac{x-a}{b}
    \right).
\]

\paragraph{Linear Tranformation of Normally Distributed Random Variable}
Suppose that \(X \sum \Normal(\mu, \sigma^2)\)
and \(a \in \mathbb{R}\) and \(b > 0\).
Then, the random variable \(Y \coloneqq a + bX\) is normally
distributed as
\[
    Y\sim \Normal(a + b\mu, b^2 \sigma^2).
\]
That is, normally distributed random variable are closed
under linear transformation.

\paragraph{Indicator Function}
An indicator function of a set \(A\) is defined by \[ 
    I_A(X) = \begin{cases}
        1 & x \in A \\
        0 & x \not\in A
    \end{cases}
.\]
Commonly, we may see indicator functions over intervals that follow the notation
\(I_{[a, b]}\).

The indicator unifies expectation and probability since, the probability
is the expectation of the indicator function.
Therefore, it may be written that \[
    \mathbb{P}(X\in A) = \int_{A} f_X(x)dx
    = \int_{-\infty}^{\infty} I_A(x) f_X(x) dx = \mathbb{E}(I_A(X))
.\]


