\section{Hypothesis Testing}

\subsection{Hypothesis}

\paragraph{Null and Alternative Hypothesis}
\begin{itemize}
  \item \textbf{Null Hypothesis:} Labelled as \(H_0\), this is a claim that, a parameter
    of interest  \(\theta\), takes a values \(\theta_0\). Hence,
      \(H_0\) is of the form  \(\theta = \theta_0\) for some pre-defined
      value of \(\theta_0\).
  \item \textbf{Alternative Hypothesis:} Labelled as \(H_1\),
    this is a more general hypothesis about  \(\theta\) that we accept as true if the
    evidence contradicting the Null Hypothesis is strong enough. \(H_1\) is usually
    of the following forms:
     \begin{itemize}
      \item \(H_1: \theta \neq  \theta_0\),
      \item \(H_1: \theta <  \theta_0\),
      \item \(H_1: \theta >  \theta_0\).
    \end{itemize}
\end{itemize}

\paragraph{Process of Hypothesis Testing}
We take the following steps.
\begin{enumerate}
  \item State the null hypothesis \((H_0)\) and the alternative hypothesis  \((H_1)\).
    By convention, the null-hypothesis is more specific.
  \item Use data to find evidence against null-hypothesis. Common ways is
     \begin{itemize}
       \item Find a statistic to measure how \textit{far} the sata is from what is expected
         if \(H_0\) is true. The approximate distribution, assuming  \(H_0\)
         is true, is called teh \textit{null distribution}.
       \item  Calculate a \(P\)-value. That is, a probability that measures the amount
         of evidence against the null-hypothesis given the observed data.
         The  \(P\)-value is the probability of observing a test statistic
         value more as more or less unususual as the one obseverd, if  \(H_0\)
         is true.
       \item Reach a conclusion about how much evidence there is against \(H_0\).
    \end{itemize}
\end{enumerate}

% \subsection{P-Values}

