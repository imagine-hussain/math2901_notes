
\section{Introduction}  

\subsection{Experiments, Sample Space and Events}

\paragraph{Experiments}
An experiment is  any process that leads to a recorded observation.

\paragraph {Outcome and Sample Space}
An outcome is possible result of the experiment. The set of all possible
outcomes is called the sample space. The sample space is often denoted by \(\Omega\).

Observe that not all sample spaces are countable. An uncountable example would be the set of all real number between \(0\) and \(1\).

\paragraph {Events}
An event is a set of outcomes that is, a subset of the sample space \(\Omega\).

\paragraph {Mutual Exclusion}
Events \(A, B\) are mutually exclusive (disjoint) if they have no outcomes in common.
That is, \(A\cap B = \emptyset\).

\paragraph {Set Operation Revision}
If you have trouble recalling the following laws, for associativity and
distributivity, you may replace \(\cap\) with \(\times\) and \(\cup\) with
\(+\). 

TODO: Associative and Distributive Law

\subsection{Sigma Algebra}
The \(\sigma\) algebra must be defined for rigorously working with probability.
The formalization of this, is beyond the scope of this course.

The \(\sigma\)-algebra can be thought of as the family of all possible
subsets or events in a sample space. Analogously, this may be
conceptualised as the power-set of the sample space.

\paragraph {Probability}
The probability is a set function, often denoted by \(\mathcal{P}\) that maps events from
the \(\sigma\)-algebra to \([0, 1]\) and satisfies certain properties.

\paragraph{Probability Space}
The triplet \(\Omega, \mathcal{A}, \mathbb{P}\) is the probability space where
\begin{itemize}
    \item \(\Omega\) is the sample space,
    \item \(\mathcal{A}\) is the \(\sigma\)-algebra,
    \item \(\mathbb{P}\) is the probability function.
\end{itemize}

\paragraph {Properties of Probability}
Given the probability/sample space \(\Omega, \mathcal{A}, \mathbb{P}\), the probability
function \(\mathbb{P}\) must satisfy
\begin{itemize}
    \item For all set \(A\in \mathcal{A}\), \(\mathbb{P}(A) \geq 0\)
    \item \(\mathbb{P}(\Omega) = 1\) 
    \item Countable additive. Suppose that the family of set \(A_i\) % TODO:
\end{itemize}

\paragraph {Theorem: Continuity from below}
Given an increasing sequence of events \(A_1 \subset A_2 \subset \dots \subset A_n\)
then, \[
    \mathbb{P}\left(
        \bigcup_{n=1}^{\infty} A_n
    \right)
    =
    \lim_{n\to\infty} = \mathbb{P} (A_n)
\]

\paragraph {Theorem: Continuity from below}
Given a decreasing sequence of events \(A_1 \supset A_2 \supset \dots A_n\)
then, \[
    \mathbb{P}\left(
        \bigcap_{n=1}^{\infty} A_n
    \right)
    =
    \lim_{n\to\infty} = \mathbb{P} (A_n)
\]

\paragraph{More Probability Lemmas}
\begin{itemize}
    \item \(\mathbb{P} (\emptyset) = 0\),
    \item For any \(A\in\mathcal{A}\), \(\mathbb{P}(A) \leq 1\) and \(\mathbb{P}(A^c) = 1-\mathbb{P}(A)\),
    \item Suppose \(A, B\in\mathcal{A}\) and \(A\subseteq B\) then \(\mathbb{P}(A) \leq \mathbb{P}(B) \).
\end{itemize}

\subsection{Conditional Probability and Independence}

\paragraph{Conditional Probability}
The conditional probability that an event \(A\) occurs given that the event
\(B\) has already occurred is denoted by 
\[
    \mathbb{P} (A|B) = \frac{\mathbb{P} (A \cap B)}{\mathbb{P} (B)}
\]

\paragraph{Independence}
The events \(A\) and \(B\) are independent if
\(
    \mathbb{P}(A\cap B) = \mathbb{P}(A)\mathbb{P}(B).
\)

\paragraph{A lemma on independence}
Given two events \(A, B\), then 
\[
    \mathbb{P}(A | B) = A
    \quad \text{ if and only if } \quad
    \mathbb{P}(B | A) = B
\]

\paragraph{Pairwise Independence of Sequences}
A countable sequence of events \(A_{i\in\mathbb{N}}\) is pairwise independent
if  \[
    \mathbb{P}(A_i0 \cap A_j) = \mathbb{P} (A_i) \mathbb{P}(A_j) \quad
    \forall i\neq j.
\]

\paragraph{Independence of Sequences}
A countable sequence of events \(A_{i\in\mathbb{N}}\) is independent is
independent if for any sub-collection \(A_{i_1},\dots, A_{i_n}\) we have
\[
    \mathbb{P}(A_{i_1} \cap A_{i_2}\cap\dots\cap A_{i_n})
    \prod_{j=1}^{n} \mathbb{P}(A_{i_j}).
\]

\paragraph{Multiplicative Law}
Given \(A, B\) are events, then,
\[
    \mathbb{P}(A \cap B) = \mathbb{P}(A | B) \mathbb{P}(B)
\]

This is equivalent to the multiplication down a decision tree.

\paragraph{Additive Law}
Let \(A, B\) be events. Then, \[
    \mathbb{P}(A\cup B) = \mathbb{P}(A) + \mathbb{P}(B) - \mathbb{P}(A \cap B).
\]
This is analogous to the inclusion-exclusion principle from set theory.


\paragraph{Law of Total Probability}
Suppose that \((A_i)_{i = 1, \dots, k}\) are mutually exclusive and
exhaustive of \(\Omega\).
That is, \[
    \bigcup_{i=1}^k A_i = \Omega.
\]

Then for any event \(B\), we have
\[
    \mathbb{P}(B) = \sum_{i=1}^k \mathbb{P}(B | A_i) \mathbb{P}(A_i).
\]
