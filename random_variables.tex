
\section{Random Variables}

\subsection{Random Variables}

\paragraph{Definition: Random Variables}
Suppose that we work on a probability space \((\Omega, \mathcal{A}, \mathbb{P}\)
And the outcomes in \(\Omega\) are denoted by \(\omega\).

Then, a random variable (r.v) \(X\) is a function from \(\Omega\) to
\(\mathbb{R}\) such that \(\forall x\in\mathbb{R}\),
the set \(
    A_x = \{\omega\in \Omega, X(\omega) \leq x\}
\).
That is, a random variable is a function that maps \(Omega\) to some space.

\paragraph{Convention on Random Variables}
Random variables are often denoted by capital letters while, the
outcomes are denoted by the lower-case equivalent of the random variable.

\paragraph{Cumulative Distributive}
The cumulative distribution of a r.v \(X\) is defined by
\[
    F_X(x) = \mathbb{P}(\{\omega: X(\omega \leq x)\}) = \mathbb{P}(X\leq x).
\]

\paragraph{Cumulative Distribution Theorems}
Suppose that \(F_X\) is cumulative distribution function of \(X\).
Then,
\begin{itemize}
    \item It is bounded between zero and one and
    \[
        \lim_{x \downarrow -\infty } = 0
        \quad\text{ and }
        \lim_{x\uparrow\infty} = 1.
    \]
    \item It is non-decreasing. That is, if \(x\leq y\) then,
    \(F_X(x) \leq F_X(y)\).
    \item For any \(x \leq y\),
    \[
        \mathbb{P}(x < X < y) = \mathbb{P}(X \leq y) - \mathbb{P}(X\leq x)
        = F_X(y) - F_X(x).
    \]
    \item It is right continuous. That is,
    \[
        \lim{x\uparrow\infty} F_X\left(x + \frac{1}{n}\right) = F_X(x).
    \]
    \item it has finite left-hand limit and
    \[
        \mathbb{P}(X< x)
        =
        \lim_{n\to\infty} F_x\left(x - \frac{1}{n}\right),
    \]
    denoted by \(F_X(x-)\).
    It is useful to observe that,
    \[
        \mathbb{P}(X=x) = F_X(X) - F_X(x-) \coloneqq F_X(x).
    \]
\end{itemize}

\paragraph{Discrete Random Variables}
A r.v. is said to be discrete if the image of \(X\) consists of 
countable many values \(x\) where \(\mathbb{P}(X = x) > 0\).
The probability function is \(\Delta F_X(x) = \mathbb{P}(X = x)\)
and satisfies
\[
    \sum_{\text{all} x} \mathbb{P} (X = x) = 1.
\]

\paragraph{Continuous Random Variables and Probability Density Functions}
A r.v is continuous if the image of \(X\) takes a continuum of values.

The probability density function of a r.v is a real-valued function
\(f_x\) on \(\mathbb{R} with\) the property that
\[
    \mathbb{P}(X \in A) = \int_{A} f_x(y)dy,
\]
for any \textit{Borel} subset of \(\mathbb{R}\)

\paragraph{Required Properties of a Density Function}
Valid density functions \(f: \mathbb{R}\to \mathbb{R}\)
must satisfy the following properties:
\begin{itemize}
    \item \(f(x) \geq 0, \forall x\in \mathbb{R}\)
    \item \(\int_{-\infty}^{\infty} f(x)dx = 1\).
\end{itemize}

\paragraph{Useful Properties of a Continuous Random Variable}
For all continuous random variables \(X\), with density \(f_x\),
\begin{enumerate}
    \item If \(A = (-\infty, x]\) and creating a cumulative distribution
    function \(F_x\) such that \(
        F_X(x) = \mathbb{P}(X\in A) = \mathbb{P}(X \leq x)
    \) then, \[
        F_X(x) = \int_{-infty}^{x} f_x(y)dy.
    \]
    \item For all \(a < b\),
    \[
        \mathbb{P}(a < X \leq b) = F_X(b) - F_X(a)
        = \int_(a)^{b} f_X(x)dx.
    \]
    \item By the fundamental theorem of calculus and property 1,
    \[
        F'_X(x) = \frac{d}{dx} \int_{-\infty}^{x} f_x(y)dy = f_X(x).
    \]
\end{enumerate}

\subsection{Expectation and Variance}

\paragraph{Expectation}
The expectation of a r.v \(X\), denoted by \(\mathbb{E}(X)\) may be
computed depending on when \(X\) is discrete or continuous.
\subparagraph{Expectation of Discrete Random Variables}
If \(X\) is a discrete random variable then,
\[
    \mathbb{E}(X) \coloneqq
    \sum_{\text{all } x} x \mathbb{P}(X = x)
    =
    \sum_{\text{all } x} x\Delta F_x(x).
\]
\subparagraph{Expectation of continuous Random Variables}
If \(X\) is a continuous random variable then,
\[
    \mathbb{E}(X) \coloneqq
    \int_{-infty}^{infty} x f_X(x)dx
\]

\paragraph{Interpreting the Expectation}
Often \(\mathbb{E}(x)\) is called the \textit{mean} of \(X\).
Observe that mean and average are not neccesarily the same.
\(\mathbb{E}(X)\) may be thought as the long-run average of
the outcomes of \(X\). That is, the average observation of 
\(X\) converges to \(\mathbb{E}(X)\).

Where our density function represents a physical model, \(\mathbb{E}(X)\)
is equivalent to the center of mass.

% TODO #4 - expecation of transformation

\paragraph{Linearity of the Expectation}
We note that the expectation is linear. That is, for all constants
\(a, b \in \mathbb{R}\),
\[
    \mathbb{E}(aX + b) = a \mathbb{E}(X) + b.
\]

\paragraph{Variance}
Let \(X\) be a r.v and set \(\mu = \mathbb{E}(x)\).
Then,
\[
    \Var(X) \coloneqq
    \mathbb{E}\left(
        \left(X - \mu \right)^2
    \right).
\]
The standard deviation is the square root of variance.

\paragraph{Properties of Variance}
Given a random variance \(X\) then, for any constants
\(a, b \in \mathbb{R}\),
\begin{enumerate}
    \item \(\Var(X) = \mathbb{E}(X^2) - \left(\mathbb{E}(X)\right)^2.\),
    \item \(\Var(aX) = a^2 \Var(X)\),
    \item \(\Var(X + b) = \Var(X)\),
    \item \(\Var(b) = 0\).
\end{enumerate}

\subsection{Moment Generating Functions}

\paragraph{Momnets}
A moment of the random variable is denoted by
\[ 
    \mathbb{E}[X^r], \quad r = 1, 2, \dots
\]

Moments measure mean, variance, skewness, and kurtosis, all ways of
looking at the shape of the distribition.

Suppose that \(f(x)\) is a probability density function. Then,
\[
    \mathbb{E}[X^r] = \int_{-\infty}^{\infty} x^r f(x)dx
\]


\paragraph{Kurtosis}
The kurtosis is the standards 4th moment. It measures how \textit{fat}
the tail is. A positive kurtosis implies a thinner tail than negative
kurtosis.

\paragraph{Moment Generating Function}
A moment generating function (MGF) is denoted as
\[
    M_x(u) = \mathbb{E}(e^{uX})
    = \int_{\text{all } x} e^{uX} f_X(x)dx.
\]
We say that the MGF of \(X\) exists if \(M_X(u)\) is finite
in some interval containing zero.

\paragraph{Using Moment Generating Function to Find Moments}
Suppose that the moment genrating funciton exists. Then,
\[
    \mathbb{E}(X^r)
    = \lim_{u\to 0} M_X^(r)(u)
    \eqqcolon \lim_{x\to 0} \frac{d^r}{d u} M_x(u) .
\]
